\hypertarget{file_formats_constraint-formats}{}\section{File formats for Secondary Structure Constraints}\label{file_formats_constraint-formats}
\hypertarget{file_formats_constraint-formats-file}{}\subsection{Constraints Definition File}\label{file_formats_constraint-formats-file}
The R\+N\+Alib can parse and apply data from constraint definition text files, where each constraint is given as a line of whitespace delimited commands. The syntax we use extends the one used in \href{http://mfold.rna.albany.edu/?q=mfold}{\tt mfold} / \href{http://mfold.rna.albany.edu/?q=DINAMelt/software}{\tt U\+N\+Afold} where each line begins with a command character followed by a set of positions.~\newline
Additionally, we introduce several new commands, and allow for an optional loop type context specifier in form of a sequence of characters, and an orientation flag that enables one to force a nucleotide to pair upstream, or downstream.\hypertarget{file_formats_const_file_commands}{}\subsubsection{Constraint commands}\label{file_formats_const_file_commands}
The following set of commands is recognized\+:
\begin{DoxyItemize}
\item {\ttfamily F} $ \ldots $ Force
\item {\ttfamily P} $ \ldots $ Prohibit
\item {\ttfamily C} $ \ldots $ Conflicts/\+Context dependency
\item {\ttfamily A} $ \ldots $ Allow (for non-\/canonical pairs)
\item {\ttfamily E} $ \ldots $ Soft constraints for unpaired position(s), or base pair(s)
\end{DoxyItemize}\hypertarget{file_formats_const_file_loop_types}{}\subsubsection{Specification of the loop type context}\label{file_formats_const_file_loop_types}
The optional loop type context specifier {\ttfamily }\mbox{[}W\+H\+E\+R\+E\mbox{]} may be a combination of the following\+:
\begin{DoxyItemize}
\item {\ttfamily E} $ \ldots $ Exterior loop
\item {\ttfamily H} $ \ldots $ Hairpin loop
\item {\ttfamily I} $ \ldots $ Interior loop (enclosing pair)
\item {\ttfamily i} $ \ldots $ Interior loop (enclosed pair)
\item {\ttfamily M} $ \ldots $ Multibranch loop (enclosing pair)
\item {\ttfamily m} $ \ldots $ Multibranch loop (enclosed pair)
\item {\ttfamily A} $ \ldots $ All loops
\end{DoxyItemize}

If no {\ttfamily }\mbox{[}W\+H\+E\+R\+E\mbox{]} flags are set, all contexts are considered (equivalent to {\ttfamily A} )\hypertarget{file_formats_const_file_orientation}{}\subsubsection{Controlling the orientation of base pairing}\label{file_formats_const_file_orientation}
For particular nucleotides that are forced to pair, the following {\ttfamily }\mbox{[}O\+R\+I\+E\+N\+T\+A\+T\+I\+O\+N\mbox{]} flags may be used\+:
\begin{DoxyItemize}
\item {\ttfamily U} $ \ldots $ Upstream
\item {\ttfamily D} $ \ldots $ Downstream
\end{DoxyItemize}

If no {\ttfamily }\mbox{[}O\+R\+I\+E\+N\+T\+A\+T\+I\+O\+N\mbox{]} flag is set, both directions are considered.\hypertarget{file_formats_const_file_seq_coords}{}\subsubsection{Sequence coordinates}\label{file_formats_const_file_seq_coords}
Sequence positions of nucleotides/base pairs are $ 1- $ based and consist of three positions $ i $, $ j $, and $ k $. Alternativly, four positions may be provided as a pair of two position ranges $ [i:j] $, and $ [k:l] $ using the \textquotesingle{}-\/\textquotesingle{} sign as delimiter within each range, i.\+e. $ i-j $, and $ k-l $.\hypertarget{file_formats_const_file_syntax}{}\subsubsection{Valid constraint commands}\label{file_formats_const_file_syntax}
Below are resulting general cases that are considered {\itshape valid} constraints\+:


\begin{DoxyEnumerate}
\item {\bfseries \char`\"{}\+Forcing a range of nucleotide positions to be paired\char`\"{}}\+:~\newline
 Syntax\+:
\begin{DoxyCode}
F i 0 k [WHERE] [ORIENTATION] 
\end{DoxyCode}
~\newline
 Description\+:~\newline
 Enforces the set of $ k $ consecutive nucleotides starting at position $ i $ to be paired. The optional loop type specifier {\ttfamily }\mbox{[}W\+H\+E\+R\+E\mbox{]} allows to force them to appear as closing/enclosed pairs of certain types of loops.
\item {\bfseries \char`\"{}\+Forcing a set of consecutive base pairs to form\char`\"{}}\+:~\newline
 Syntax\+:\begin{DoxyVerb}F i j k [WHERE] \end{DoxyVerb}
~\newline
 Description\+:~\newline
 Enforces the base pairs $ (i,j), \ldots, (i+(k-1), j-(k-1)) $ to form. The optional loop type specifier {\ttfamily }\mbox{[}W\+H\+E\+R\+E\mbox{]} allows to specify in which loop context the base pair must appear.
\item {\bfseries \char`\"{}\+Prohibiting a range of nucleotide positions to be paired\char`\"{}}\+:~\newline
 Syntax\+:\begin{DoxyVerb}P i 0 k [WHERE] \end{DoxyVerb}
~\newline
 Description\+:~\newline
 Prohibit a set of $ k $ consecutive nucleotides to participate in base pairing, i.\+e. make these positions unpaired. The optional loop type specifier {\ttfamily }\mbox{[}W\+H\+E\+R\+E\mbox{]} allows to force the nucleotides to appear within the loop of specific types.
\item {\bfseries \char`\"{}\+Probibiting a set of consecutive base pairs to form\char`\"{}}\+:~\newline
 Syntax\+:\begin{DoxyVerb}P i j k [WHERE] \end{DoxyVerb}
~\newline
 Description\+:~\newline
 Probibit the base pairs $ (i,j), \ldots, (i+(k-1), j-(k-1)) $ to form. The optional loop type specifier {\ttfamily }\mbox{[}W\+H\+E\+R\+E\mbox{]} allows to specify the type of loop they are disallowed to be the closing or an enclosed pair of.
\item {\bfseries \char`\"{}\+Prohibiting two ranges of nucleotides to pair with each other\char`\"{}}\+:~\newline
 Syntax\+:\begin{DoxyVerb}P i-j k-l [WHERE] \end{DoxyVerb}
 Description\+:~\newline
 Prohibit any nucleotide $ p \in [i:j] $ to pair with any other nucleotide $ q \in [k:l] $. The optional loop type specifier {\ttfamily }\mbox{[}W\+H\+E\+R\+E\mbox{]} allows to specify the type of loop they are disallowed to be the closing or an enclosed pair of.
\item {\bfseries \char`\"{}\+Enforce a loop context for a range of nucleotide positions\char`\"{}}\+:~\newline
 Syntax\+:\begin{DoxyVerb}C i 0 k [WHERE] \end{DoxyVerb}
 Description\+:~\newline
 This command enforces nucleotides to be unpaired similar to {\itshape prohibiting} nucleotides to be paired, as described above. It too marks the corresponding nucleotides to be unpaired, however, the {\ttfamily }\mbox{[}W\+H\+E\+R\+E\mbox{]} flag can be used to enforce specfic loop types the nucleotides must appear in.
\item {\bfseries \char`\"{}\+Remove pairs that conflict with a set of consecutive base pairs\char`\"{}}\+:~\newline
 Syntax\+:\begin{DoxyVerb}C i j k \end{DoxyVerb}
~\newline
 Description\+:~\newline
 Remove all base pairs that conflict with a set of consecutive base pairs $ (i,j), \ldots, (i+(k-1), j-(k-1)) $. Two base pairs $ (i,j) $ and $ (p,q) $ conflict with each other if $ i < p < j < q $, or $ p < i < q < j $.
\item {\bfseries \char`\"{}\+Allow a set of consecutive (non-\/canonical) base pairs to form\char`\"{}}\+:~\newline
 Syntax\+:
\begin{DoxyCode}
A i j k [WHERE] 
\end{DoxyCode}
~\newline
 Description\+:~\newline
 This command enables the formation of the consecutive base pairs $ (i,j), \ldots, (i+(k-1), j-(k-1)) $, no matter if they are {\itshape canonical}, or {\itshape non-\/canonical}. In contrast to the above {\ttfamily F} and {\ttfamily W} commands, which remove conflicting base pairs, the {\ttfamily A} command does not. Therefore, it may be used to allow {\itshape non-\/canoncial} base pair interactions. Since the R\+N\+Alib does not contain free energy contributions $ E_{ij} $ for non-\/canonical base pairs $ (i,j) $, they are scored as the {\itshape maximum} of similar, known contributions. In terms of a {\itshape Nussinov} like scoring function the free energy of non-\/canonical base pairs is therefore estimated as \[ E_{ij} = \min \left[ \max_{(i,k) \in \{GC, CG, AU, UA, GU, UG\}} E_{ik}, \max_{(k,j) \in \{GC, CG, AU, UA, GU, UG\}} E_{kj} \right]. \] The optional loop type specifier {\ttfamily }\mbox{[}W\+H\+E\+R\+E\mbox{]} allows to specify in which loop context the base pair may appear.
\item {\bfseries \char`\"{}\+Apply pseudo free energy to a range of unpaired nucleotide positions\char`\"{}}\+:~\newline
 Syntax\+:
\begin{DoxyCode}
E i 0 k e 
\end{DoxyCode}
~\newline
 Description\+:~\newline
 Use this command to apply a pseudo free energy of $ e $ to the set of $ k $ consecutive nucleotides, starting at position $ i $. The pseudo free energy is applied only if these nucleotides are considered unpaired in the recursions, or evaluations, and is expected to be given in $ kcal / mol $.
\item {\bfseries \char`\"{}\+Apply pseudo free energy to a set of consecutive base pairs\char`\"{}}\+:~\newline
 Syntax
\begin{DoxyCode}
E i j k e 
\end{DoxyCode}
~\newline
 Use this command to apply a pseudo free energy of $ e $ to the set of base pairs $ (i,j), \ldots, (i+(k-1), j-(k-1)) $. Energies are expected to be given in $ kcal / mol $. 
\end{DoxyEnumerate}